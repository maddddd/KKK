\documentclass{VUMIFPSkursinis}
\usepackage{algorithmicx}
\usepackage{algorithm}
\usepackage{algpseudocode}
\usepackage{amsfonts}
\usepackage{amsmath}
\usepackage{bm}
\usepackage{caption}
\usepackage{color}
\usepackage{float}
\usepackage{graphicx}
\usepackage{listings}
\usepackage{subfig}
\usepackage{wrapfig}

% Titulinio aprašas
\university{Vilniaus universitetas}
\faculty{Matematikos ir informatikos fakultetas}
\department{Programų sistemų katedra}
\papertype{Laboratorinis darbas nr. 1}
\title{Knygų keitimosi klubas}
\titleineng{}
\status{2 kurso 5 grupės studentai}
\author{Vidmantas Bakštys, Tadas Petrauskas}
\secondauthor{Tadas Vaitiekūnas} % Pavardes pagal abecele :)
% \secondauthor{Vardonis Pavardonis}   % Pridėti antrą autorių
\supervisor{lekt. dr. Vytautas Valaitis}
\date{Vilnius – 2017}

% Nustatymai
% \setmainfont{Palemonas}   % Pakeisti teksto šriftą į Palemonas (turi būti įdiegtas sistemoje)
\bibliography{bibliografija}

\begin{document}
\maketitle

\sectionnonum{Anotacija}
Laboratoriniame darbe apibrėžiami reikalavimai sistemai, struktūrinis dalykinės srities modelis (angl. Domain modelling), preliminarios sistemos atliekamos užduotys (angl. Use case modelling) ir atliekama reikalavimų peržiūra. Darbas atliekamas taikant ICONIX procesą.\\
Darbą įgyvendina Programų sistemų 2 kurso 5 grupės studentai Vidmantas Bakštys, Tadas Petrauskas, Tadas Vaitiekūnas.

\setcounter{tocdepth}{2}

\tableofcontents

\sectionnonum{Įvadas}
Darbo tikslas - susipažinimas su ICONIX proceso principais. 

% Citavimo pavyzdžiai: cituojamas vienas šaltinis \cite{PvzStraipsnLt}; cituojami
% keli šaltiniai \cite{PvzStraipsnEn, PvzKonfLt, PvzKonfEn, PvzKnygLt, PvzKnygEn,
% PvzElPubLt, PvzElPubEn, PvzMagistrLt, PvzPhdEn}.

\section{Reikalavimai}

\section{Struktūrinis dalykinės srities modelis}

\section{Preliminarios sistemos atliekamos užduotys}

\section{Reikalavimų peržiūra}









\end{document}
