\documentclass{VUMIFPSkursinis}
\usepackage{algorithmicx}
\usepackage{algorithm}
\usepackage{algpseudocode}
\usepackage{amsfonts}
\usepackage{amsmath}
\usepackage{bm}
\usepackage{caption}
\usepackage{color}
\usepackage{float}
\usepackage{graphicx}
\usepackage{listings}
\usepackage{subfig}
\usepackage{wrapfig}
\usepackage{array}

% Allow having thick borders in a table
%\makeatletter
%\newcommand{\thickhline}{%
%    \noalign {\ifnum 0=`}\fi \hrule height 2pt
%    \futurelet \reserved@a \@xhline
%}
%\newcolumntype{"}{@{\vrule width 2pt}}
%\makeatother

% Titulinio aprašas
\university{Vilniaus universitetas}
\faculty{Matematikos ir informatikos fakultetas}
\department{Programų sistemų katedra}
\papertype{Architekturos dokumentacija}
\title{Knygų keitimosi klubas}
\titleineng{}
\status{2 kurso 5 grupės studentai}
\author{Vidmantas Bakštys}
\secondauthor{Tadas Petrauskas}
\thirdauthor{Tadas Vaitiekūnas} % Pavardes pagal abecele :)
\supervisor{Vytautas Valaitis}
\date{Vilnius – 2016}

% Nustatymai
% \setmainfont{Palemonas}   % Pakeisti teksto šriftą į Palemonas (turi būti įdiegtas sistemoje)
\bibliography{bibliografija}

\begin{document}
\maketitle

\sectionnonum{Anotacija}
%Šio darbo tikslas yra suprojektuoti ir aprašyti busimos programų sistemos - Automobilių detalių prekybos platforma, architektūrą.
%Kuriant šį darbą buvo remiamasi ankščiau atlikta užsakovo poreikių bei dalykinės srities analize, bei reikalavimų specifikacija.
%Darbą įgyvendina Programų sistemų 2 kurso 5 grupės studentai Vidmantas Bakštys, Tadas Petrauskas, Vytautas Saulis, Tadas Vaitiekūnas.

\setcounter{tocdepth}{2}

\tableofcontents

\setcounter{secnumdepth}{0}

\sectionnonum{Įvadas}
%Šiame dokumente pateikiama programų sistemos „Automobilių dalių prekybos platforma" (toliau ADPP) architektūra. Ji yra apibrėžiama aprašant sistemą keliais aspektais (pjūviais), naudojant 4+1 pjūvių modelį. Architektūra buvo projektuojama taip, kad būtų įgyvendinami ankščiau pateikti užsakovo poreikiai, bei remiantis dalykinės srities analize. Šis dokumentas yra skirtas programuotojams, kuriantiems šią sistemą, bei būsimos sistemos administratoriams.
%
\setcounter{secnumdepth}{4}

\section{Reikalavimai}

\setcounter{secnumdepth}{0}
\sectionnonum{Priedai}
\subsection{Užsakovo reikalavimai sistemai}
\begin{enumerate}
	\item Leisti vartotojui prisijungti prie sistemos naudojant socialinių tinklų paskyras;
	\item Leisti vartotojui užregistruoti knyga įvedant ISBN kodą;
	\item Kiekvieną kartą vartotjui užregistruojant naują knygą, pagal ISBN koda patikrinti
		ar knyga egzistuoja. Jei ne, knygos neužregistruoti;
	\item Vartotojui užregistruojant naują knygą leisti ieškoti knygos pagal 
		nebūtinai pilną pavadinimą arba autoriu;
	\item Leisti ieškoti knygos sistemoje pagal ISBN kodą;
	\item Kiekvienam vartotojui leisti susikurti savo norimų gauti knygų sarašą (wishlist);
	\item Suteikti galimybe vartotojui pasiūlyti savo įkeltą knygą kitam vartotojui;
	\item Suteikti galimybe vartotojui ieškoti kitų vartotjų, bei susisiekti su jais ne vien
		perkant iš jų knygas;
	\item Vartotojo užregistruotos knygos puslapyje automatiškai pridėti knygos aprašymą
		iš interneto;
	\item Leisti pridėti savo komentarus prie savo įkeltos knygos (apie knygos kokybę,
		pačią knyga ir panašiai);
	\item Vartotojo paskyroje rodyt išsamią vartotojo statistiką (sėkmingų mainų skaičius,
		įvertinimai);
\end{enumerate}


\setcounter{secnumdepth}{4}


%\printbibliography[heading=bibintoc]  % Šaltinių sąraše nurodoma panaudota
\sectionnonum{Šaltinių sarašas}
\begin{itemize}
\item http://www.mif.vu.lt/~karolis/PSI1.html
\item http://www.mif.vu.lt/katedros/se/Studentams/KURSINIO%20DARBO%20METODINIAI%20NURODYMAI202011
\item http://www.autoplus.lt
\item http://www.latex-project.org/help/documentation/
\end{itemize}

% literatūra, kitokie šaltiniai. Abėcėlės tvarka išdėstomi darbe panaudotų
% (cituotų, perfrazuotų ar bent paminėtų) mokslo leidinių, kitokių publikacijų
% bibliografiniai aprašai.  Šaltinių sąrašas spausdinamas iš naujo puslapio.
% Aprašai pateikiami netransliteruoti. Šaltinių sąraše negali būti tokių
% šaltinių, kurie nebuvo paminėti tekste.

\end{document}
