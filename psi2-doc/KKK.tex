\documentclass{VUMIFPSkursinis}
\usepackage{algorithmicx}
\usepackage{algorithm}
\usepackage{algpseudocode}
\usepackage{amsfonts}
\usepackage{amsmath}
\usepackage{bm}
\usepackage{caption}
\usepackage{color}
\usepackage{float}
\usepackage{graphicx}
\usepackage{listings}
\usepackage{subfig}
\usepackage{wrapfig}
\usepackage{array}
\usepackage{enumitem}

% Allow having thick borders in a table
%\makeatletter
%\newcommand{\thickhline}{%
%    \noalign {\ifnum 0=`}\fi \hrule height 2pt
%    \futurelet \reserved@a \@xhline
%}
%\newcolumntype{"}{@{\vrule width 2pt}}
%\makeatother

% Titulinio aprašas
\university{Vilniaus universitetas}
\faculty{Matematikos ir informatikos fakultetas}
\department{Programų sistemų katedra}
\papertype{Reikalavimų specifikacija}
\title{Knygų keitimosi klubas}
\titleineng{}
\status{2 kurso 5 grupės studentai}
\author{Vidmantas Bakštys}
\secondauthor{Tadas Petrauskas}
\thirdauthor{Tadas Vaitiekūnas} % Pavardes pagal abecele :)
\supervisor{Vytautas Valaitis}
\date{Vilnius – 2016}

% Nustatymai
% \setmainfont{Palemonas}   % Pakeisti teksto šriftą į Palemonas (turi būti įdiegtas sistemoje)
\bibliography{bibliografija}

\begin{document}
\maketitle

\setcounter{tocdepth}{2}

\pagenumbering{gobble}  % nerodyti puslapio numberio turinyje
\tableofcontents
\pagenumbering{arabic} 

\setcounter{secnumdepth}{0}

\sectionnonum{Anotacija}
%Šio darbo tikslas yra suprojektuoti ir aprašyti busimos programų sistemos - Automobilių detalių prekybos platforma, architektūrą.
%Kuriant šį darbą buvo remiamasi ankščiau atlikta užsakovo poreikių bei dalykinės srities analize, bei reikalavimų specifikacija.
%Darbą įgyvendina Programų sistemų 2 kurso 5 grupės studentai Vidmantas Bakštys, Tadas Petrauskas, Vytautas Saulis, Tadas Vaitiekūnas.


\sectionnonum{Įvadas}
%Šiame dokumente pateikiama programų sistemos „Automobilių dalių prekybos platforma" (toliau ADPP) architektūra. Ji yra apibrėžiama aprašant sistemą keliais aspektais (pjūviais), naudojant 4+1 pjūvių modelį. Architektūra buvo projektuojama taip, kad būtų įgyvendinami ankščiau pateikti užsakovo poreikiai, bei remiantis dalykinės srities analize. Šis dokumentas yra skirtas programuotojams, kuriantiems šią sistemą, bei būsimos sistemos administratoriams.
%
\setcounter{secnumdepth}{4}

\section{Reikalavimai}
	Šiame skyriuje pateikiami funkciniai bei nefunkciniai reikalavimai sistemai, apibrėžiantys norimą sistemos pokytį.
	Reikalavimai buvo sudaryti remiantis esama sistema ir užsakavo pateiktais pradiniais reikalavimais,
	kurie yra pateikiami šio dokumento priede.
\subsection{Funkciniai reikalavimai}
	\begin{enumerate}[label=\textbf{FR\arabic*}]
		\item Užsiregistravęs vartotojas gali prisijungti:
			\begin{enumerate}[label*=\textbf{.\arabic*}]
				\item Naudodamasis savo vartotojo vardu (arba elektroniniu paštu) ir slaptažodžiu;
				\item Naudodamasis socialinio tinklo paskyrą;
			\end{enumerate}
		\item Norimų gauti knygų sąrašas;
			\begin{enumerate}[label*=\textbf{.\arabic*}]
				\item Kiekvienas registruotas vartotojas gali pridėti knygų prie savo sarašo;
				\item Vartotojas gali įkelti knygą į savo saraša nurodydams ISBN kodą;
				\item Vartotojas gali ieškoti knygos internete pagal pavadinimą ar autorių ir 
					pasirinkęs ją įkelti į savo sarašą;
				\item Sistema neleidžia įkelti neegzistuojančios knygos į savo sarašą;
				\item Vartotojai gali matyti kitų vartotojų sarašus;
				\item Varotojui pranešti (el paštu) kada sistemoje atsiranda knyga esenti jo saraše;
		    \end{enumerate}
		\item Knygos įkėlimas į sistemą;
			\begin{enumerate}[label*=\textbf{.\arabic*}]
				\item Vartotojas gali įkelti knygą įvesdamas knygos ISBN kodą;
				\item Vartotojas gali ieškoti knygos pagal pavadinimą ar autorių ir 
					pasirinkęs ją įkelti;
				\item Sistemą leidžia vartotojui įkelti knygą tik jei knyga su tuo ISBN kodu egzistuoja;
				\item Vartotojas gali pridėti komentarą apie savo įkeltą knygą;
				\item Vartotojo pridėtos knygos puslapyje automatiškai pridėti oficialų aprašymą apie knygą;
			\end{enumerate}
		\item Įkeltų knygų paieška;
			\begin{enumerate}[label*=\textbf{.\arabic*}]
				\item Lankytojas gali ieškoti knygos sistemoje pagal ISBN kodą;
				\item Lankytojas gali ieškoti knygos pagal raktinius žodžius;
				\item Jei knygos sistemoje nėra, ieškoti jos internete, radus pasiūlyti pridėti
					prie norimų gauti knygų sarašo;
				\item Knygos gali ieškoti ir neregistruotas lankytojas;
			\end{enumerate}
		\item Vartotojai gali ieškoti kitų vartotojų, matyti jų puslapius ir su jais susisiekti;
		\item Vartotojas gali pasiūlyti savo įkeltą knygą kitam vartotojui;
		\item Vartotojo puslapyje rodyti išsamią statistiką;
			\begin{enumerate}[label*=\textbf{.\arabic*}]
				\item Sėkmingų mainų skaičius;
				\item Įkeltų knygų skaičius; 
				\item Kitų vartotojų įvertinimai;
				\item Ivertinimų vidurkis;
			\end{enumerate}
	\end{enumerate}
\subsection{Nefunkciniai reikalavimai}
	\begin{enumerate}[label=\textbf{NFR\arabic*}]
		\item Tinklapis turi būti pasiekiamas ir patogus naudoti, per mobilųjį įrenginį;
		\item Turi būti užtikrintas duomenų vientisumas duomenų bazėje;
		\item Turi būti užtikrinta galimybė atkurti duomenų bazės duomenis, įvykus sutrikimams;
		\item Turi būti sukurta sistemos administravimo dokumentacija;
	\end{enumerate}

\section{Struktūrinis dalykinės srities modelis}
	Šiame skyriuje pateikiamas struktūrinis nagrinėjamos dalykinės srities modelis. 
	Modelis pateikiamas UML klasių diagramomis kartu su žodynu - sąrašu esybių su jų aprašymais. 
	\subsection{Esybių diagrama}
		\begin{figure}[H]
			\centering
			\includegraphics[scale=0.7]{img/DomainModel.png}
			\caption{Dalykinės srities esybės}
			\label{img:psi2}
		\end{figure}

	\subsection{Žodynas}
		\begin{enumerate}[label=\textbf{E\arabic*.}]
			\item Vartotojas - Asmuo užsiregistravęs (turintis paskyrą) sistemoje;
			\item Lankytojas - Asmuo apsilankęs tinklapyje (nebūtinai vartotojas);
			\item Knyga - literatūros kūrinys, turintis ISBN kodą;
			\item Įkelta knyga - į KKK sistemą ikelta knyga;
			\item Knygų duomenų bazė - Duomenų bazė prie kurios sistema jungsis ieškodama knygų, tikrindama ISBN kodus;
			\item Knygų sąrašas - bet kokių knygų rinkinys;
			\item Ikeltų knygų katalogas - sąrašas visų į KKK sistemą įkeltų knygų;
			\item Knygos aprašymas - oficialus knygos aprašymas paimtas iš knygų duomenų bazės;
			\item Komentaras - Vartotojo komentaras apie savo įkeltą knygą;
			\item Pageidavimų sąrašas - Sarašas knygų, kurias vartotjas norėtų įsigyti;
			\item Vartotojo puslapis - vartotojo turimas paskyros puslapis kuriame patalpinta visa viešai matoma vartotojo informacija;
			\item Paieškos rezultatas - knygos paieškos knygų duomenų bazėje rezultatas, ar KKK sistemoje rezultatas;
			\item Vartotojo statistika - Statistinė informacija apie vartotojo paskyrą: sėkmingų mainų skaičius, įkeltų knygų skaičius,
				kitų vartotojų įvertinimai, įvertinimų vidurkis;
		\end{enumerate}

\section{Užduotys}
	Šiame skyriuje pateikiamos sistemos atliekamos užduotys, jų pagrindiniai, bei alternatyvūs scenarijai.
	%TODO: uzduociu uml
	\begin{enumerate}[label=\textbf{U\arabic*.}]
		%TODO: gui eskizai
		\item Užsiregistruoti;		% leisti asocijuoti su soc. tinklo paskyra
		\item Prisijungti;
		\item Pridėti knygą prie pageidavimų sąrašo;
		\item Įkelti knygą į sistemą;
		\item Ieškoti knygos įkeltų knygų kataloge;
		\item Susisiekti su kitais vartotojais;
		\item Susisiekti su pardevėju;
		\item Susisiekti su pirkėju;
		\item Įvertinti mainus; % fiksuoti statistiką
		\item Pasiūlyti savo įkeltą knygą kitam vartotojui;
	\end{enumerate}

\section{Peržiūros metu rastos klaidos}
	\begin{itemize}
		\item 
	\end{itemize}

	
\setcounter{secnumdepth}{0}
\sectionnonum{Priedai}
\subsection{Užsakovo reikalavimai sistemai}
\begin{enumerate}
	\item Leisti vartotojui prisijungti prie sistemos naudojant socialinių tinklų paskyras;
	\item Leisti vartotojui užregistruoti knyga įvedant ISBN kodą;
	\item Kiekvieną kartą vartotjui užregistruojant naują knygą, pagal ISBN koda patikrinti
		ar knyga egzistuoja. Jei ne, knygos neužregistruoti;
	\item Vartotojui užregistruojant naują knygą leisti ieškoti knygos pagal 
		nebūtinai pilną pavadinimą arba autoriu;
	\item Leisti ieškoti knygos sistemoje pagal ISBN kodą;
	\item Kiekvienam vartotojui leisti susikurti savo norimų gauti knygų sarašą (wishlist);
	\item Neradus ieškomos knygos sistemoje, pasiūlyti ją pridėti prie norimų gauti knygų sarašo;
	\item Suteikti galimybe vartotojui pasiūlyti savo įkeltą knygą kitam vartotojui;
	\item Suteikti galimybe vartotojui ieškoti kitų vartotojų, bei susisiekti su jais ne vien
		perkant iš jų knygas;
	\item Vartotojo užregistruotos knygos puslapyje automatiškai pridėti knygos aprašymą
		iš interneto;
	\item Leisti pridėti savo komentarus prie savo įkeltos knygos (apie knygos kokybę,
		pačią knyga ir panašiai);
	\item Vartotojo paskyroje rodyt išsamią vartotojo statistiką (sėkmingų mainų skaičius,
		įvertinimai);
	\item Tinklapis turi būti pasiekiamas ir patogus naudoti, per mobilųjį įrenginį;
	\item Turi būti užtikrintas duomenų vientisumas duomenų bazėje;
	\item Turi būti užtikrinta galimybė atkurti duomenų bazės duomenis, įvykus sutrikimams;
	\item Turi būti sukurta sistemos administravimo dokumentacija;
\end{enumerate}


\setcounter{secnumdepth}{4}


%\printbibliography[heading=bibintoc]  % Šaltinių sąraše nurodoma panaudota
\sectionnonum{Šaltinių sąrašas}
	\begin{itemize}
		\item http://www.mif.vu.lt/~karolis/PSI2.html
	\end{itemize}

% literatūra, kitokie šaltiniai. Abėcėlės tvarka išdėstomi darbe panaudotų
% (cituotų, perfrazuotų ar bent paminėtų) mokslo leidinių, kitokių publikacijų
% bibliografiniai aprašai.  Šaltinių sąrašas spausdinamas iš naujo puslapio.
% Aprašai pateikiami netransliteruoti. Šaltinių sąraše negali būti tokių
% šaltinių, kurie nebuvo paminėti tekste.

\end{document}
